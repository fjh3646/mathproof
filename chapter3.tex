\chapter{中值定理与凹凸性}
\section{罗尔定理相关问题}
%1
\begin{example}
	\color{red}证明广义罗尔定理:设$f(x)$在$x\geqslant a$时连续,在$x>a$时可导,又设$f(a)=\lim_{x\to+\infty}f(x)$,则存在$\xi\in(a,+\infty)$使得$f'(\xi)=0$\color{black}。
\end{example}

	\begin{newproof}
			若$f\left( x \right) \equiv C$,则一定$\exists \xi \in \left( a,+\infty \right) $,使$f'\left( \xi \right) =0$\\
		若$f(x)$不恒为常数,设\[\left( a \right) =\underset{x\rightarrow +\infty}{\lim}f\left( x \right) =A\]
		则一定存在\[x_0\in \left( a,+\infty \right) \]
		使\[f\left( x_0 \right) >A\text{或}f\left( x_0 \right) <A\]
		不妨设\[ f\left( x_0 \right) >A\left( f\left( x_0 \right) <A\text{时类似可证} \right)  \]
		由广义介值定理,因\[ f\left( x \right) \text{在}\left[ a,\left. +\infty \right) \right. \text{上连续} \]
		故\[ \forall M\in \left( A,f\left( x_0 \right) \right) ,\exists x_1\in \left( a,x_0 \right) ,x_2\in \left( x_0,+\infty \right)  \]
		使得\[ f\left( x_1 \right) =f\left( x_2 \right) =M\]
		因\[ f\left( x \right) \text{在}\left[ x_1,x_2 \right] \text{连续,}\left( x_1,x_2 \right) \text{可导} \]
		由罗尔定理\[ \exists \xi \in \left( x_1,x_2 \right) \subset \left( a,+\infty \right)  \]
		使得\[ f'\left( \xi \right) =0 \]
		
	\end{newproof}

%2
\begin{example}
	设$f(x)$在$[0,1]$上连续,在$(0,1)$内可导,且$f(0)=f(1)=0$,试证明:若存在$x_0\in(0,1)$使得$f(x_0)>x_0$,则必然存在$\xi\in(0,1)$使得$f'(\xi)=1$。
\end{example}
	\begin{newproof}
		
	\end{newproof}

\begin{example}
	假设$f(x)$和$g(x)$在$[a,b]$上存在二阶导数,并且$g''(x)\neq 0$,$f(a)=f(b)=g(a)=g(b)=0$,试证明
	\begin{enumerate}
		\item 在$(a,b)$内$g(x)=\neq 0$;
		\item 在$(a,b)$内至少有一点$\xi$使得$\frac{f(\xi)}{g(\xi)}=\frac{f''(\xi)}{g''(\xi)}$。
	\end{enumerate}
\end{example}

\begin{example}
	设$f(x)$在$[0,+\infty)$上可导,且$0\leqslant f(x)\leqslant\frac{x}{1+x^2}$,试证明$\exists\xi\in(0,+\infty)$是$f'(\xi)=\frac{1-\xi^2}{(1+\xi^2)^2}$。
\end{example}

\begin{example}
	设$f(x)$在$[a,b]$上连续,在$(a,b)$内可导,试证明$\exists\xi(a,b)$,使得\[f'(\xi)=\frac{f(\xi)-f(a)}{b-\xi}\]
\end{example}

\begin{example}
	设$f(x)$在$[a,b]$上连,在$(a,b)$内可导,试证明$\exists\xi\in(a,b)$使得\[\frac{bf(b)-af(a)}{b-a}=f(\xi)+\xi f'(\xi)\]
\end{example}

\begin{example}
	设$f(x)$,$g(x)$都在$[a,b]$上连续,在$(a,b)$内可导,且$g(x)\neq=0$,$f(a)g(b)=g(a)f(b)$,试证明至少存在一点$\xi\in(a,b)$,使得$f'(\xi)g(\xi)=f(\xi)g'(\xi)$。
\end{example}

\begin{example}
	设$f(x)$,$g(x)$在$(-\infty,+\infty)$内可导,且对一切$x$都有$f'(x)g(x)\neq f(x)g'(x)$,证明方程$f(x)=0$的任何两个不同的根之间必有$g(x)=0$的根。
\end{example}

\begin{example}
	设$f(x)$在$[0,a]$上连续,在$(0,a)$内可导,$f(a)=0$证明$\forall\alpha\in(0,+\infty)$,总存在$\xi\in(0,a)$,使得$\alpha f(\xi)+\xi f'(\xi)=0$。
\end{example}

\begin{example}
	设$f(x)$在$[a,b]$上连续,在$(a,b)$内可导,$\lambda$为实数,$f(a)=f(b)=0$,证明$\exists\xi\in(a,b)$,使得$\lambda f(\xi)+f'(\xi)=0$。
\end{example}

\begin{example}
	设$f(x)$在$[a,b]$上连续,在$(a,b)$内可导,且有$f(a)=f(b)=0$,$g(x)$也在$[a,b]$上连续,$(a,b)$内可导,求证$\exists\xi\in(a,b)$,使得$f'(\xi)+g'(\xi)f(\xi)=0$。
\end{example}

\begin{example}
	设$f(x)$在$[a,b]$上连续,$(a,b)$内可导,其中$a>0$,且有$f(a)=0$,证明$\exists\xi\in(a,b)$,使得$f(\xi)=\frac{b-\xi}{a}f'(\xi)$。
\end{example}

\begin{example}
	设$f(x)$在$[0,1]$上连续,在$(0,1)$内可微,且$f(0)=f(1)=0$,$f(\frac{1}{2})=1$,证明
	\begin{enumerate}
		\item $\exists\xi\in(\frac{1}{2},1)$,使得$f(\xi)=\xi$;
		\item 存在一个$\eta\in(0,\xi)$,使得$f'(\eta)=f(\eta)-\eta+1$。
	\end{enumerate}
\end{example}

\begin{example}
	设$f(x)$,$g(x)$均在$[1,6]$上连续,在$(1,6)$内可导,且$f(1)=5$,$f(5)=1$,$f(6)=12$,求证$\exists\xi\in(1,6)$,使得$f'(\xi)+g'(\xi)[f(\xi)-2\xi]=2$。
\end{example}

\begin{example}
	设$f(x)$在$[0,1]$上连续,在$(0,1)$内可导,$f(0)=0$,$分f(x)$在$(0,1)$内非零,试证明在$(0,1)$内至少存在一点$\xi$,使得$\frac{f'(\xi)}{f(\xi)}=\frac{f'(1-\xi)}{f(1-\xi)}$。
\end{example}
\begin{example}
	设$f(x)$在$[0,1]$上连续,在$(0,1)$内可导,$f(0)=0$,$f(x)$在$(0,1)$内非零,试证在$(0,1)$内至少存在一点$\xi$,使得$\frac{mf'(\xi)}{f(\xi)}=\frac{nf'(1-\xi)}{f(1-\xi)}$。
\end{example}

\begin{example}
	设$f(x)$在$[0,4]$上可导,且$f(0)=0$,$f(1)=1$,$f(4)=2$,证明至少存在一点$\xi\in(0,4)$,使得$f''(\xi)=-\frac{1}{3}$。
\end{example}

\begin{example}
	设奇函数$f(x)$在$[-1,1]$上具有二阶导数,且$f(1)=1$,证明
	\begin{enumerate}
		\item 存在一点$\xi\in(0,1)$,使得$f'(\xi)=1$;
		\item 存在一点$\eta\in(-1,1)$,使得$f''(\eta)+f'(\eta)=1$。
	\end{enumerate}
\end{example}

\begin{example}
	设$f(x)$在$[a,b]$上连续,在$(a,b)$内可导,且$f(a)f(b)>0$,$f(a)f(\frac{a+b}{2})<0$,证明对任何实数$k$,必定存在$\xi\in(a,b)$,使得$f'(\xi)=kf(\xi)$。
\end{example}

\begin{example}
	设$f(\xi)$在闭区间$[a,b]$上连续,在$(a,b)$内可导,且$f(a)=f(b)=\frac{a}{2}$,$f(\frac{a+b}{2})=a+b$,其中$0<a<b$,证明对任意的$\lambda$,存在$\xi\in(a,b)$,使得\[f'(\xi)=\lambda\left[f(\xi)-\frac{1}{2}\xi\right]+\frac{1}{2}\]
\end{example}

\begin{example}
	设$f(x)$在$[0,1]$上具有二阶导数,且$f'(0)=0$,试证明存在$\xi\in(0,1)$,使$f''(\xi)=\frac{2f'(\xi)}{1-\xi}$。
\end{example}

\begin{example}
	设$f(x)$在$[0,1]$上具有二阶导数,且$f(0)=f(1)=0$,试证明存在$\xi\in(0,1)$,使$f''(\xi)=\frac{2f'(\xi)}{1-\xi}$。
\end{example}

\begin{example}
	设$f(x)$在$[a,b]$上具有二阶导数,$f(0)=f(1)=f'(0)=f'(1)$,证明存在$\xi\in(0,1)$,使得$f''(\xi)=f(\xi)$。
\end{example}

\begin{example}
	设$f'(x)$在$[a,b]$上连续,$f(x)$在$(a,b)$内二阶可导,$f(a)=f(b)=0$,且有$\int_{a}^{b}f(x)\diff x=0$,证明
	\begin{enumerate}
		\item 在$(a,b)$内至少存在一点$\xi$,使得$f'(\xi)=f(\xi)$;
		\item 在$(a,b)$内至少存在一点$\eta$,$\eta\neq\xi$,使得$f''(\eta)=f(\eta)$。
	\end{enumerate}
\end{example}

\begin{example}
	设$f(x)$,$g(x)$在$[a,b]$上连续,在$(a,b)$内具有二阶导数,且存在相等的最大值,又$f(a)=g(a)$,$f(b)=g(b)$,证明存在$\xi\in(a,b)$,使得$f''(\xi)=g''(\xi)$。
\end{example}

\begin{example}
	设$a_1$,$a_2$,$\cdots$,$a_n$为$n$个实数,并满足$a_1-\frac{a_2}{3}+\cdots+(-1)^n\frac{a_n}{2n-1}=0$,证明方程$a_1\cos x+a_2\cos 3x+\cdots+a_n\cos (2n-1)x=0$在$(0,\frac{\pi}{2})$至少有一实根。
\end{example}

\begin{example}
	设$\frac{a_0}{n+1}+\frac{a_1}{n}+\cdots+\frac{a_{n-1}}{2}+a_n=0$,证明方程$a_0x^n+a_1x^{n-1}+\cdots+a_n=0$至少有一个不小于$1$的实根。
\end{example}

\begin{example}
	证明方程$2^x-x^2=1$有且仅有三个实根。
\end{example}

\begin{example}
	若$f(x)$为可导函数,$g(x)$为连续函数,试证明在$f(x)$的两个零点之间,一定有$f'(x)-kf(x)g(x)=0$的零点。
\end{example}

\begin{example}
	设$f(x)$在$[0,1]$上连续,在$(0,1)$内可导,且满足$f(1)=k\int_{a}^{b}xe^{1-x}f(x)\diff x(k>1)$,证明至少存在一点$\xi\in(0,1)$,使得$f'(\xi)=(1-\frac{1}{\xi})f(\xi)$。
\end{example}

\begin{example}
	设$f(x)$,$g(x)$在$[a,b]$上连续,试证明存在$\xi\in(a,b)$,使得\[f(\xi)\int_{\xi}^{b}g(x)\diff x=g(\xi)\int_{a}^{\xi}f(x)\diff x\]
\end{example}

\begin{example}
	已知函数$f(x)$在$[0,1]$上连续,在$(0,1)$内二阶可导,且$f(1)=0$,设函数$g(x)=x^2f(x)$,证明至少存在一点$\xi\in(0,1)$,使得$g''(\xi)=0$。
\end{example}

\begin{example}
	设$f(x)$在$[0,1]$上二阶可导,且$f(1)>0$,$\lim_{x\to 0^+}\frac{f(x)}{x}<0$,证明
	\begin{enumerate}
		\item 方程$f(x)=0$在区间$(0,1)$内至少存在一个实根;
		\item 方程$f(x)f''(x)+[f'(x)]^2=0$在区间$(0,1)$内至少存在两个不同的实根。
	\end{enumerate}
\end{example}

\begin{example}
	已知函数$f(x)$在$[a,b]$上连续,在$(a,b)$内二阶可导,且$f(a)=f(b)$,$f'(x)\neq 0$,证明$\exists\xi\in(a,b)$,使得$[f'(\xi)]^2=f(\xi)f''(\xi)$。
\end{example}

\begin{example}
	$f(x)$在$[a,b]$上连续,在$(a,b)$内二阶可导,且$f(a)=f(b)=\int_{a}^{b}f(x)\diff x=0$,证明
	\begin{enumerate}
		\item $\exists\xi_1\in(a,b)$,使得$f''(\xi_1)=f(\xi_1)$;
		\item $\exists\xi_2\in(a,b)$,使得$f''(\xi_2)-3f'(\xi_2)+2f(\xi_2)=0$。
	\end{enumerate}
\end{example}

\begin{example}
	设$f(x)$在$[0,\pi]$上连续,在$(0,\pi)$内二阶可导,且$f(0)+f(\pi)=0$,证明$\exists\xi\in(0,\pi)$,使得$f''(\xi)+f(\xi)=0$。
\end{example}

\begin{example}
	设$f(x)$在$[0,+\infty)$上有连续导数,$f(0)=1$,且对一切$x\geqslant0$有$|f(x)|\leqslant e^{-x}$,证明存在一点$\xi\in(0,+\infty)$,使得$f'(\xi)=-e^{-\xi}$。
\end{example}

\begin{example}
	设$f(x)$在$[0,1]$上二阶可导,且$f(0)=f(1)=0$,证明至少存在一点$\xi\in(0,1)$,使得$\xi^2f''(\xi)+4\xi f(\xi)=2f(\xi)=0$。
\end{example}

\begin{example}
	设函数$f(x)$具有二阶导数,且$f(0)=0$,证明存在$\xi\in\left(-\frac{\pi}{2},\frac{\pi}{2}\right)$,使得\[f''(\xi)=f(\xi)(1+\tan^2\xi)\]
\end{example}

\section{k值法}

\begin{example}
	设$f(x)$在$[a,b]$上二阶可导,$f(a)=f(b)=0$,证明对每个$x\in(a,b)$,存在$\xi\in(a,b)$,使得$f(x)=\frac{f''(\xi)}{2}(x-a)(x-b)$。
\end{example}

\begin{example}
	已知函数$f(x)$在$[0,1]$上三阶可导,且$f(0)=-1$,$f(1)=0$,$f'(0)=0$,证明存在一点$\xi\in(0,1)$,使得$f(x)=-1+x^2+\frac{x^2(x-1)}{3!}f'''(\xi)x\in(0,1)$。
\end{example}

\begin{example}
	设$f(x)$在$(0,1)$内有三阶导数,$0<a<b<1$,证明存在$\xi\in(a,b)$,使得$f(b)=f(a)+\frac{1}{2}(b-a)[f'(a)+f'(b)]-\frac{(b-a)^3}{12}f'''(\xi)$。
\end{example}

\begin{example}
	设$f(x)$在$a\leqslant x\leqslant b$上连续,在$(a,b)$内二阶可导,证明在$a<x<b$上有\[\frac{\frac{f(x)-f(a)}{x-a}-\frac{f(b)-f(a)}{b-a}}{x-b}=\frac{1}{2}f''(\xi)\quad(a<\xi<b)\]
\end{example}

\begin{example}
	设$f(x)$在$[a,b]$上具有连续的二阶导数,求证$\exists\in(a,b)$,使得\[\int_{a}^{b}f(x)\diff x=(b-a)f(\frac{a+b}{2})+\frac{1}{24}(b-a)^3f'''(\xi)\]
\end{example}

\begin{example}
	设$f(x)$在$[a,b]$上二阶可导,求证至少存在一点$\xi\in(a,b)$,使得\[\int_{a}^{b}f(x)\diff x=(b-a)\frac{f(a)+f(b)}{2}-\frac{1}{12}f''(\xi)(b-a)^3\]
\end{example}

\begin{example}
	设$f(x)$在$[a,b]$上连续,在$(a,b)$内二阶可导,求证$\exists\xi\in(a,b)$,使得\[f(b)-2f(\frac{a+b}{2})+f(a)=\frac{(b-a)^2}{4}f''(\xi)\]
\end{example}

\begin{example}
	设$f(x)$在$[a,b]$上可导,在$(a,b)$内二阶可导,若$a<c<b$,证明存在$\xi\in(a,b)$,使得$\frac{f(a)}{(a-b)(a-c)}+\frac{f(b)}{(b-a)(b-c)}+\frac{f(c)}{(c-a)(c-b)}=\frac{1}{2}f''(\xi)$。
\end{example}

\begin{example}
	设有实数$a_1$,$a_2$,$\cdots$,$a_n$,其中$a_1<a_2<\cdots<a_n$,函数$f(x)$在$[a_1,a_n]$上具有$n$阶导数,并满足$f(a_1)=f(a_2)=\cdots=f(a_n)=0$,证明对任意的$c\in[a_1,a_n]$,都相应的有$\xi\in(a_1,a_n)$,使得$f(c)=\frac{(c-a_1)(c-a_2)\cdots(c-a_n)}{n!}f^{(n)}(\xi)$。
\end{example}

\section{拉格朗日中值定理——弦线法}
\begin{example}
	设不恒为零的函数$f(x)$在$[a,b]$上连续,在$(a,b)$内可导,且$f(a)=f(b)$,证明在$(a,b)$内至少存在一点$\xi$,使得$f'(\xi)>0$。
\end{example}

\begin{example}
	设$f(x)$在$[0,1]$上连续,在$(0,1)$内可导,且$f(0)=0$,证明如果$f(x)$在$(0,1)$内不恒等于零,则必定存在一点$\xi\in(0,1)$,使得$f'(\xi)f(\xi)>0$。
\end{example}

\begin{example}
	设$f(x)$在$[a,b]$上连续,在$(a,b)$内具有二阶导数,且$f(a)=f(b)=0$,$f(c)<0(a<c<b)$,证明至少存在一点$\xi\in(a,b)$,使得$f''(\xi)>0$。
\end{example}

\begin{example}
	设$f(x)$在$[a,b]$上连续,在$(a,b)$内二阶可导,连接点$A(a,f(a))$,$B(b,f(b))$的直线段$AB$与曲线$y=f(x)$相交于点$C(c,f(c))(a<c<b)$,证明$\exists\in\xi(a,b)$使得$f''(\xi)=0$。
\end{example}

\begin{example}
	设$f(X)$在$[a,b]$上连续,在$(a,b)$内可导,又$f(x)$不是线性函数,且$f(b)>f(a)$,试证明$\exists\xi\in(a,b)$,使得$f'(\xi)>\frac{f(b)-f(a)}{b-a}$。
\end{example}

\begin{example}
	\color{red}设$f(X)$在$[0,1]$上可微,$f(0)=0$,$f(1)=1$,$k_1$,$k_2$,$\cdots$,$k_n$为$n$个正数,证明在$[0,1]$内存在一组互不相等的数$x_1$,$x_2$,$\cdots$,$x_n$使得$\sum_{i=1}^n\frac{k_i}{f'(x_i)}=\sum_{i=1}^nk_i$\color{black}。
\end{example}

\begin{example}
	\color{red}已知$f(x)$在$[0,1]$上连续,在$(0,1)$内可导,且$f(0)=0$,$f(1)=1$,证明
	\begin{enumerate}
		\item 存在$\xi\in(0,1)$,使得$f(\xi)=1-\xi$\color{black};
		\color{red}\item 存在两个不同的$\alpha$,$\beta\in(0,1)$,使得$f'(\alpha)f'(\beta)=1$\color{black}。
	\end{enumerate}
\end{example}

\begin{example}
	\color{red}设$f(x)$在$[0,1]$上连续,在$(0,1)$内可导,且$f(0)=0$,$f(1)=1$,试证明对于任意给定的正数$a$,$b$,在$(0,1)$内存在不同的点$\xi$和$\eta$使得$\frac{a}{f'(\xi)}+\frac{b}{f'(\eta)}=a+b$\color{black}。
\end{example}

\begin{example}
	设$f(x)$在$[a,b]$上可导,$f(0)=0$,$f(1)=1$,试证明在区间$[0,1]$上存在两个不同的点$x_1$,$x_2$使得$\frac{1}{f'(x_1)}+\frac{1}{f'(x_2)}=2$。
\end{example}

\begin{example}
	已知$f(x)$在$[0,1]$上连续,在$(0,1)$内可导,且$f(0)=0$,$f(1)=1$,证明存在互不相等的$\xi_1$,$\xi_2$,$\cdots$,$\xi_8\in(0,1)$,使得对正的常数$\ln H$,$\ln A^2$,$\ln E^2$,$\ln P^2$,$\ln Y^2$,$\ln R$,$\ln W$,$\ln N$满足下式\[e^{\frac{\ln H}{f'(\xi_1)}+\frac{\ln N}{f'(\xi_2)}+\frac{\ln W}{f'(\xi_3)}+\frac{\ln R}{f'(\xi_4)}+\frac{\ln A^2}{f'(\xi_5)}+\frac{\ln P^2}{f'(\xi_6)}+\frac{\ln Y^2}{f'(\xi_7)}+\frac{\ln E^2}{f'(\xi_8)}}=HAPPY\cdot NEW\cdot YEAR\]
\end{example}

\begin{example}
	设$f(x)$在$[0,1]$上连续,在$(0,1)$内可导,且$f(0)=0$,$f(1)=1$,$a$,$b$为给定的正数,证明$\exists\xi$,$\eta$,$0<\xi<\eta<1$,使得$af'(\xi)+bf'(\eta)=a+b$。
\end{example}

\section{拉格朗日中值定理——作为函数的表达}

\begin{example}
	设$f(x)$在$[0,+\infty)$上可微,且$f(0)=0$,并设有实数$A>0$,使得$|f'(x)|\leqslant A|f(x)|$,在$[0,+\infty)$成立,试证明在$[0,+\infty)$上$f(x)\equiv 0$。
\end{example}

\begin{example}
	设$[0,a]$上$|f''(x)|\leqslant M$,$f(x)$在$(0,a)$内取最大值,试证明\[|f'(0)|+|f'(a)|\leqslant Ma\]
\end{example}

\begin{example}
	证明若函数$f(x)$在$(0,+\infty)$内可微,且$\lim_{n\to+\infty}f'(x)=0$,则$\lim_{n\to+\infty}\frac{f(x)}{x}=0$。
\end{example}

\begin{example}
	设$f(x)$在$[a,+\infty)$上可导,且当$x>a$时,$f'(x)>k>0$,其中$k为常数$,证明若$f(a)<0$,则方程$f(x)=0$在$[a,a-\frac{f(a)}{k}]$内有且仅有一个实根。
\end{example}

\begin{example}
	设$f(x)$在有限区间$(a,b)$内可导,且$f'(x)$在该区间内有界,证明
	\begin{enumerate}
		\item $f(x)$在$(a,b)$内有界;
		\item $\lim_{x\to a^+}f(x)$与$\lim_{x\to b^-}f(x)$均存在。
	\end{enumerate}
\end{example}

\begin{example}
	证明若函数$f(x)$在开区间$(a,b)$内可导且无界,则$f'(x)$在$(a,b)$内也无界。
\end{example}

\begin{example}
	\begin{enumerate}
		\item 设$f(x)$在$(a,+\infty)$内可导,则$\lim_{x\to+\infty}f(x)$与$\lim_{x\to+\infty}f'(x)$都存在,证明$\lim_{x\to+\infty}f'(x)=0$;
		\item 设$f(x)$在$(-\infty,+\infty)$内可导,且$\lim_{x\to\infty}f(x)$与$\lim_{x\to\infty}f'(x)$都存在,证明$\lim_{x\to\infty}f'(x)=0$。
	\end{enumerate}
\end{example}

\begin{example}
	设$f(x)$在$(a,+\infty)$内可导,且$\lim_{x\to+\infty}f'(x)=A>0$,证明$\lim_{x\to+\infty}f(x)=+\infty$。
\end{example}

\begin{example}
	设$f(x)$在$[0,+\infty)$上可导,且当$x>0$时,$a<f'(x)<\frac{1}{x^2}$,证明$\lim_{x\to+\infty}f(x)$存在。
\end{example}

\begin{example}
	设$f(x)$在$[0,1]$上有连续导数,且$f(0)=f(1)=0$,证明$|\int_{a}^{b}f(x)\diff x|\leqslant\frac{M}{4}$,其中$M$是$|f'(x)|$在$[0,1]$上的最大值。
\end{example}

\begin{example}
	设$f(x)$在$[0,1]$上有连续导数,且$f(0)=0$,证明$|\int_{a}^{b}f(x)\diff x|\leqslant\frac{M}{2}$,其中$M$是$|f'(x)|$在$[0,1]$上的最大值。
\end{example}

\begin{example}
	设$f(x)$在$[0,1]$上有连续导数,且$f(a)=f(b)=0$,证明$|\int_{a}^{b}f(x)\diff x|\leqslant\frac{(b-a)^2M}{4}$,其中$M$是$|f'(x)|$在$[0,1]$上的最大值。
\end{example}

\begin{example}
	设$f(x)$在$(-\infty,+\infty)$上二次可微且有界,试证明存在$x_0\in(-\infty,+\infty)$,使得$f''(x_0)=0$。
\end{example}

\begin{example}
	设$f(x)$在$(-\infty,+\infty)$上具有二阶导数,且$f''(x)>0$,$\lim_{x\to+\infty}f'(x)=\alpha>0$,$\lim_{x\to-\infty}f'(x)=\beta<0$,又存在一点$x_0$,使得$f(x_0)<0$,试证明方程$f(x)=0$在$(-\infty,+\infty)$上有且只有两个实根。
\end{example}
