\chapter{中值定理与凹凸性}
\section{罗尔定理相关问题}
%1
\begin{example}
	\color{red}证明广义罗尔定理:设$f(x)$在$x\geqslant a$时连续,在$x>a$时可导,又设$f(a)=\lim_{x\to+\infty}f(x)$,则存在$\xi\in(a,+\infty)$使得$f'(\xi)=0$\color{black}。
\end{example}

	\begin{newproof}
			若$f\left( x \right) \equiv C$,则一定$\exists \xi \in \left( a,+\infty \right) $,使$f'\left( \xi \right) =0$\\
		若$f(x)$不恒为常数,设\[\left( a \right) =\underset{x\rightarrow +\infty}{\lim}f\left( x \right) =A\]
		则一定存在\[x_0\in \left( a,+\infty \right) \]
		使\[f\left( x_0 \right) >A\text{或}f\left( x_0 \right) <A\]
		不妨设\[ f\left( x_0 \right) >A\left( f\left( x_0 \right) <A\text{时类似可证} \right)  \]
		由广义介值定理,因\[ f\left( x \right) \text{在}\left[ a,\left. +\infty \right) \right. \text{上连续} \]
		故\[ \forall M\in \left( A,f\left( x_0 \right) \right) ,\exists x_1\in \left( a,x_0 \right) ,x_2\in \left( x_0,+\infty \right)  \]
		使得\[ f\left( x_1 \right) =f\left( x_2 \right) =M\]
		因\[ f\left( x \right) \text{在}\left[ x_1,x_2 \right] \text{连续,}\left( x_1,x_2 \right) \text{可导} \]
		由罗尔定理\[ \exists \xi \in \left( x_1,x_2 \right) \subset \left( a,+\infty \right)  \]
		使得\[ f'\left( \xi \right) =0 \]
		
	\end{newproof}

%2
\begin{example}
	设$f(x)$在$[0,1]$上连续,在$(0,1)$内可导,且$f(0)=f(1)=0$,试证明:若存在$x_0\in(0,1)$使得$f(x_0)>x_0$,则必然存在$\xi\in(0,1)$使得$f'(\xi)=1$。
\end{example}
	\begin{newproof}
		令\[F\left( x \right) =f\left( x \right) -x,x\in \left[ \text{0,}1 \right] \]
		则\[F\left( x_0 \right) =f\left( x_0 \right) -x_0>\text{0,}F\left( 1 \right) =f\left( 1 \right) -1=-1<0\]
		则\[F\left( x_0 \right) F\left( 1 \right) <0\]
		又$F\left( x \right) \text{在}\left[ x_0,1 \right] \text{连续}$,由零点存在定理\[\exists a\in \left( x_0,1 \right) \]
		使得\[F\left( a \right) =0\]
		因\[F\left( 0 \right) =0=F\left( a \right) ,F\left( x \right) \text{在}\left[ \text{0,}a \right] \text{上连续,}\left( \text{0,}a \right) \text{可导}\]
		由罗尔定理\[\exists \xi \in \left( \text{0,}a \right) \subset \left( \text{0,}1 \right) \]
		使得\[F'\left( \xi \right) =0\Rightarrow f'\left( \xi \right) =1\]
		
	\end{newproof}

%3
\begin{example}
	假设$f(x)$和$g(x)$在$[a,b]$上存在二阶导数,并且$g''(x)\neq 0$,$f(a)=f(b)=g(a)=g(b)=0$,试证明
	\begin{enumerate}
		\item 在$(a,b)$内$g(x)=\neq 0$;
		\item 在$(a,b)$内至少有一点$\xi$使得$\frac{f(\xi)}{g(\xi)}=\frac{f''(\xi)}{g''(\xi)}$。
	\end{enumerate}
\end{example}
	\begin{newproof}
		\begin{enumerate}
			\item 若\[\exists x_0\in \left( a,b \right) \]
			使得\[g\left( x_0 \right) =0\]
			那么有\[g\left( a \right) =g\left( x_0 \right) =g\left( b \right) \]
			因\[g\left( x \right) \text{在}\left( a,x_0 \right) \text{连续,}\left( a,x_0 \right) \text{可导,}\left[ x_0,b \right] \text{连续,}\left( x_0,b \right) \text{可导}\]
			由罗尔定理\[\exists x_3\in \left( x_1,x_2 \right) \]
			使\[g^{''}\left( x_3 \right) =0\]
			这与$g^{''}\left( x \right) \ne 0$矛盾,故\[g\left( x \right) \ne 0\]

			\item 令\[F\left( x \right) =f\left( x \right) g'\left( x \right) -g\left( x \right) f'\left( x \right) \]
			因\[F\left( a \right) =F\left( b \right) =\text{0,}F\left( x \right) \text{在}\left[ a,b \right] \text{上连续,}\left( a,b \right) \text{内可导}\]
			故\[\exists \xi \in \left( a,b \right) \]
			使\[F'\left( \xi \right) =0\]
			即\[\frac{f\left( \xi \right)}{g\left( \xi \right)}=\frac{f^{''}\left( \xi \right)}{g^{''}\left( \xi \right)}\text{成立}\]
			
		\end{enumerate}
	\end{newproof}
	\begin{note}
	$\frac{f\left( \xi \right)}{g\left( \xi \right)}=\frac{f^{''}\left( \xi \right)}{g^{''}\left( \xi \right)}\Longleftrightarrow f\left( \xi \right) g^{''}\left( \xi \right) -g\left( \xi \right) f^{''}\left( \xi \right) =0$
	\begin{align*}
		\int{\left[ f\left( x \right) g^{''}\left( x \right) -g\left( x \right) f^{''}\left( x \right) \right]dx}
		= {}&
		\int{f\left( x \right) dg'\left( x \right) -\int{g\left( x \right) df'\left( x \right)}}\\
		={}&
		f\left( x \right) g'\left( x \right) -\int{g'\left( x \right) f'\left( x \right) dx}-g\left( x \right) f'\left( x \right) +\int{f'\left( x \right)}g'\left( x \right) dx\\
		={}&
		f\left( x \right) g'\left( x \right) -g\left( x \right) f'\left( x \right) 
	\end{align*}
	\end{note}
%4
\begin{example}
	设$f(x)$在$[0,+\infty)$上可导,且$0\leqslant f(x)\leqslant\frac{x}{1+x^2}$,试证明$\exists\xi\in(0,+\infty)$是$f'(\xi)=\frac{1-\xi^2}{(1+\xi^2)^2}$。
\end{example}
	\begin{newproof}
		令\[F\left( x \right) =f\left( x \right) -\frac{x}{1+x^2},x\in \left[ \text{0,}\left. +\infty \right) \right. \]
		因\[0\leqslant f\left( x \right) \leqslant \frac{x}{1+x^2}\text{且}\underset{x\rightarrow +\infty}{\lim}\frac{x}{1+x^2}=0\]
		由迫敛性知\[\underset{x\rightarrow +\infty}{\lim}f\left( x \right) =0\]
		即有\[\underset{x\rightarrow +\infty}{\lim}F\left( x \right) =\underset{x\rightarrow +\infty}{\lim}\left[ f\left( x \right) -\frac{x}{1+x^2} \right] =0\]
		因\[F\left( 0 \right) =f\left( 0 \right) \geqslant \text{0,}\forall x\in \left( \text{0,}+\infty \right) \]
		有\[F\left( x \right) \leqslant 0\]
		又\[F\left( x \right) \text{在}\left[ \text{0,}+\infty \right] \text{连续}\]
		由零点存在定理\[\exists x_0\in \left[ \text{0,}+\infty \right] \]
		使\[F\left( x_0 \right) =0\]
		因\[F\left( x_0 \right) =\lim_{x\rightarrow +\infty} f\left( x \right) =\text{0且}F\left( x \right) \text{在}\left[ x_0 \right. ,\left. +\infty \right) \text{连续,}\left[ x_0 \right. ,\left. +\infty \right) \text{可导}\]
		由广义罗尔定理\[\exists \xi \in \left( x_0,+\infty \right) \]
		使\[F'\left( \xi \right) =0\]
		即\[f'\left( \xi \right) =\frac{1-\xi ^2}{\left( 1+\xi ^2 \right) ^2}\]
	\end{newproof}

%5
\begin{example}
	设$f(x)$在$[a,b]$上连续,在$(a,b)$内可导,试证明$\exists\xi(a,b)$,使得\[f'(\xi)=\frac{f(\xi)-f(a)}{b-\xi}\]
\end{example}
	\begin{newproof}
		令\[F\left( x \right) =\left( b-x \right) f\left( x \right) +f\left( a \right) x\]
		有\[F\left( a \right) =\left( b-a \right) f\left( a \right) +f\left( a \right) \cdot a=bf\left( a \right) \]
		\[F\left( b \right) =F\left( a \right) b=F\left( a \right) \]
		又\[F\left( x \right) \text{在}\left[ a,b \right] \text{连续,}\left( a,b \right) \text{内可导}\]
		由罗尔定理\[\exists \xi \in \left( a,b \right) \]
		使\[F'\left( \xi \right) =0\]
		即\[f'\left( \xi \right) =\frac{f\left( \xi \right) -f\left( a \right)}{b-\xi}\]
	\end{newproof}

%6
\begin{example}
	设$f(x)$在$[a,b]$上连,在$(a,b)$内可导,试证明$\exists\xi\in(a,b)$使得\[\frac{bf(b)-af(a)}{b-a}=f(\xi)+\xi f'(\xi)\]
\end{example}
	\begin{newproof}
		\begin{itemize}
			\item[方法一] 令\[\text{令}F\left( x \right) =xf\left( x \right) \,\,  x\in \left[ a,b \right] \]
			因\[F\left( x \right) \text{在}\left[ a,b \right] \text{上连续,}\left( a,b \right) \text{内可导}\]
			由拉格朗日中值定理\[F\left( b \right) -F\left( b \right) =F'\left( \xi \right) \left( b-a \right) \text{,其中}a<\xi <b\]
			即\[\frac{bf\left( b \right) -af\left( a \right)}{b-a}=f\left( \xi \right) +\xi f'\left( \xi \right) \]
			\begin{note}
				对于$\frac{bf\left( b \right) -af\left( a \right)}{b-a}$,分子形式为$xf\left( x \right) $,而分母恰是端点$a,b$之差所以联想到拉格朗日中值定理或柯西中值定理
			\end{note}
			\item[方法二] 令\[F\left( x \right) =\frac{bf\left( b \right) -af\left( a \right)}{b-a}\left( x-a \right) -xf\left( x \right) \,\,  x\in \left[ a,b \right] \]
			有\[F\left( a \right) =-af\left( a \right) ,F\left( b \right) =-af\left( a \right) =F\left( a \right) \]
			因\[F\left( x \right) \text{在}\left[ a,b \right] \text{上连续,}\left( a,b \right) \text{内可导}\]
			由罗尔定理\[\exists \xi \in \left( a,b \right) \]
			使得\[F'\left( \xi \right) =0\Longrightarrow \frac{bf\left( b \right) -af\left( a \right)}{b-a}=f\left( \xi \right) +\xi f'\left( \xi \right) \]
			\begin{note}
				由\[\frac{bf\left( b \right) -af\left( a \right)}{b-a}=f\left( \xi \right) +\xi f'\left( \xi \right) \]
				积分得到\[\int{\frac{bf\left( b \right) -af\left( a \right)}{b-a}dx}=\int{\left[ f\left( x \right) +xf'\left( x \right) \right] dx}\]
				即\[\frac{bf\left( b \right) -af\left( a \right)}{b-a}\left( x-a \right) -xf\left( x \right) =C\]
			\end{note}
		\end{itemize}
	\end{newproof}

%7
\begin{example}
	设$f(x)$,$g(x)$都在$[a,b]$上连续,在$(a,b)$内可导,且$g(x)\neq=0$,$f(a)g(b)=g(a)f(b)$,试证明至少存在一点$\xi\in(a,b)$,使得$f'(\xi)g(\xi)=f(\xi)g'(\xi)$。
\end{example}
	\begin{newproof}
		令\[F\left( x \right) =\frac{f\left( x \right)}{g\left( x \right)}\,\, x\in \left[ a,b \right] \]
		因\[f\left( a \right) f\left( b \right) =g\left( a \right) g\left( b \right) \]
		故\[\frac{f\left( a \right)}{g\left( a \right)}=\frac{f\left( b \right)}{g\left( b \right)}\]
		即\[F\left( a \right) =F\left( b \right) \]
		因\[F\left( x \right) \text{在}\left[ a,b \right] \text{连续,}\left( a,b \right) \text{内可导}\]
		由罗尔定理\[\exists \xi \in \left( a,b \right) \]
		使\[F'\left( \xi \right) =0\]
		即\[f'\left( \xi \right) g\left( \xi \right) =f\left( \xi \right) g'\left( \xi \right) \]
	\end{newproof}
	\begin{note}
		\begin{align*}
			f'\left( \xi \right) g\left( \xi \right) =f\left( \xi \right) g'\left( \xi \right)
			\Longrightarrow {}&
			\frac{f'\left( x \right)}{f\left( x \right)}=\frac{g'\left( x \right)}{g\left( x \right)}\\
			\Longrightarrow {}&
			\int{\frac{f'\left( x \right)}{f\left( x \right)}}dx=\int{\frac{g'\left( x \right)}{g\left( x \right)}dx}\\
			\Longrightarrow {}&
			\ln \left| f\left( x \right) \right|=\ln \left| g\left( x \right) \right|+\ln\text{|}C|\\
			\Longrightarrow {}&
			\frac{f\left( x \right)}{g\left( x \right)}=C
		\end{align*}
		故构造\[F\left( x \right) =\frac{f\left( x \right)}{g\left( x \right)}\]
	\end{note}
%8
\begin{example}
	设$f(x)$,$g(x)$在$(-\infty,+\infty)$内可导,且对一切$x$都有$f'(x)g(x)\neq f(x)g'(x)$,证明方程$f(x)=0$的任何两个不同的根之间必有$g(x)=0$的根。
\end{example}
	\begin{newproof}
		设方程$f(x) = 0$的任意某两根为$x_1,x_2\left( x_1<x_2 \right) $
		则\[f\left( x_1 \right) =f\left( x_2 \right) =0\]
		[\bf $\text{只要证明:}\exists x_0\in \left[ x_1,x_2 \right] ,\text{使}g\left( x_0 \right) =\text{0即可}$]\\
		如果不存在\[x_0\in \left[ x_1,x_2 \right] \]
		使\[g\left( x \right) \not \equiv 0\]
		令\[F\left( x \right) =\frac{f\left( x \right)}{g\left( x \right)}\]
		因\[F\left( x_1 \right) =F\left( x_2 \right) =\text{0且}F\left( x \right) \text{在}\left[ x_1,x_2 \right] \text{连续,}\left( x_1,x_2 \right) \text{内可导}\]
		故由罗尔定理\[\exists \xi \in \left( x_1,x_2 \right) \]
		使\[f'\left( \xi \right) g\left( \xi \right) -g'\left( \xi \right) f\left( \xi \right) =\text{0成立}\]
		这与题设矛盾,故\[\exists x_0\in \left( x_1,x_2 \right) \]
		使\[g\left( x_0 \right) =0\]
		得证
	\end{newproof}
%9
\begin{example}
	设$f(x)$在$[0,a]$上连续,在$(0,a)$内可导,$f(a)=0$证明$\forall\alpha\in(0,+\infty)$,总存在$\xi\in(0,a)$,使得$\alpha f(\xi)+\xi f'(\xi)=0$。
\end{example}
	\begin{newproof}
		令\[F\left( x \right) =x^{\alpha}f\left( x \right) \,\,      x\in \left[ \text{0,}a \right] \]
		因\[F\left( 0 \right) =F\left( a \right) =\text{0且}F\left( x \right) \text{在}\left[ \text{0,}a \right] \text{上连续,}\left( \text{0,}a \right) \text{内可导}\]
		故由罗尔定理\[\exists \xi \in \left( \text{0,}a \right) \]
		使\[F'\left( \xi \right) =0\]
		即\[\alpha f\left( \xi \right) +\xi f'\left( \xi \right) =0\]
	\end{newproof}
	\begin{note}
		\begin{align*}
			\alpha f\left( \xi \right) +\xi f'\left( \xi \right) =0
			\Longrightarrow {}&	
			\frac{f'\left( \xi \right)}{f\left( \xi \right)}=-\frac{\alpha}{\xi}\\
			\Longrightarrow {}&	
			\int{\frac{f'\left( \xi \right)}{f\left( \xi \right)}dx}=-\int{\frac{\alpha}{x}dx}\\
			\Longrightarrow {}&	
			\ln\text{|}f\left( x \right) |=-\alpha \ln\text{|}x|+\ln\text{|}C|\\
			\Longrightarrow {}&	
			x^{\alpha}f\left( x \right) =C	
		\end{align*}
	\end{note}

%10
\begin{example}
	设$f(x)$在$[a,b]$上连续,在$(a,b)$内可导,$\lambda$为实数,$f(a)=f(b)=0$,证明$\exists\xi\in(a,b)$,使得$\lambda f(\xi)+f'(\xi)=0$。
\end{example}
	\begin{newproof}
		令\[F\left( x \right) =e^{\lambda x}f\left( x \right) \]
		因\[F\left( a \right) =F\left( b \right) =\text{0且}F\left( x \right) \text{在}\left[ a,b \right] \text{上连续,}\left( a,b \right) \text{内可导}\]
		故有罗尔定理\[\exists \xi \in \left( a,b \right) \]
		使\[F'\left( \xi \right) =0\]
		即\[e^{\lambda \xi}\left[ \lambda f\left( \xi \right) +f'\left( \xi \right) \right] =0\]
		又\[e^{\lambda \xi}>0\]
		故\[\lambda f\left( \xi \right) +f'\left( \xi \right) =0\]
		得证
	\end{newproof}
	\begin{note}
		\begin{align*}
			\lambda f\left( \xi \right) +f'\left( \xi \right) =0
			\Longrightarrow {}&
			\frac{f'\left( \xi \right)}{f\left( \xi \right)}=-\lambda \\
			\Longrightarrow {}&
			\int{\frac{f'\left( \xi \right)}{f\left( \xi \right)}dx}=\int{-\lambda dx}
			\Longrightarrow {}&
			\ln\text{|}f\left( x \right) |=-\lambda x+\ln\text{|}C|\\
			\Longrightarrow {}&
			e^{\lambda x}f\left( x \right) =C
		\end{align*}
		因此构造\[F\left( x \right) =e^{\lambda x}f\left( x \right) \]
	\end{note}
	
%11
\begin{example}
	设$f(x)$在$[a,b]$上连续,在$(a,b)$内可导,且有$f(a)=f(b)=0$,$g(x)$也在$[a,b]$上连续,$(a,b)$内可导,求证$\exists\xi\in(a,b)$,使得$f'(\xi)+g'(\xi)f(\xi)=0$。
\end{example}
	\begin{newproof}
		令\[F\left( x \right) =e^{g\left( x \right)}f\left( x \right) \]
		因\[F\left( a \right) =F\left( b \right) =\text{0且}F\left( x \right) \text{在}\left[ a,b \right] \text{上连续,}\left( a,b \right) \text{内可导}\]
		由罗尔定理\[\exists \xi \in \left( a,b \right) \]
		使\[F'\left( \xi \right) =0\]
		即\[e^{g\left( \xi \right)}\left[ f'\left( \xi \right) +g'\left( \xi \right) f\left( x \right) \right] =0\]
		故\[f'\left( \xi \right) +g'\left( \xi \right) f\left( x \right) =0\]
		得证
	\end{newproof}
	\begin{note}
		\begin{align*}
			f'\left( \xi \right) +g'\left( \xi \right) f\left( \xi \right) =0
			\Longrightarrow {}&
			\frac{f'\left( \xi \right)}{f\left( \xi \right)}=-g'\left( \xi \right) \\
			\Longrightarrow {}&
			\int{\frac{f'\left( \xi \right)}{f\left( \xi \right)}dx=-\int{g'\left( \xi \right)}}\\
			\Longrightarrow {}&
			\ln \left| f\left( x \right) \right|=-g'\left( \xi \right) +\ln C\\
			\Longrightarrow {}&
			e^{g\left( x \right)}f\left( x \right) =C
		\end{align*}
	\end{note}

%12
\begin{example}
	设$f(x)$在$[a,b]$上连续,$(a,b)$内可导,其中$a>0$,且有$f(a)=0$,证明$\exists\xi\in(a,b)$,使得$f(\xi)=\frac{b-\xi}{a}f'(\xi)$。
\end{example}
	\begin{newproof}
		令\[F\left( x \right) =\left( x-b \right) ^af\left( x \right) \,\,     x\in \left[ a,b \right] \]
		因\[F\left( a \right) =F\left( b \right) =\text{0且}F\left( x \right) \text{在}\left[ a,b \right] \text{连续,}\left( a,b \right) \text{可导}\]
		由罗尔定理\[\exists \xi \in (a,b)\]
		使得\[F'\left( \xi \right) =0\]
		即\[a\left( \xi -b \right) ^{a-1}f\left( \xi \right) +\left( \xi -b \right) ^{\alpha}f'\left( \xi \right) =0\]
		又\[\xi \ne b\]
		故约去\[\left( \xi -b \right) ^{a-1}\]
		得\[af\left( \xi \right) +\left( \xi -b \right) f'\left( \xi \right) =0\Longrightarrow f\left( \xi \right) =\frac{b-\xi}{a}f'\left( \xi \right) \]
	\end{newproof}
	\begin{note}
		\begin{align*}
			f\left( \xi \right) =\frac{b-\xi}{a}f'\left( \xi \right) 
			\Longrightarrow {}&
			\frac{f'\left( x \right)}{f\left( x \right)}=\frac{a}{b-x}\\
			\Longrightarrow {}&
			\int{\frac{f'\left( x \right)}{f\left( x \right)}}dx=\int{\frac{a}{b-x}dx}\\
			\Longrightarrow {}&
			\ln \left| f\left( x \right) \right|=-a\ln \left| x-b \right|+\ln \left| c \right|\\
			\Longrightarrow {}&
			\left( x-b \right) ^af\left( x \right) =c
		\end{align*}
		因此构造\[F\left( x \right) =\left( x-a \right) ^af\left( x \right) \]
	\end{note}

%13
\begin{example}
	设$f(x)$在$[0,1]$上连续,在$(0,1)$内可微,且$f(0)=f(1)=0$,$f(\frac{1}{2})=1$,证明
	\begin{enumerate}
		\item $\exists\xi\in(\frac{1}{2},1)$,使得$f(\xi)=\xi$;
		\item 存在一个$\eta\in(0,\xi)$,使得$f'(\eta)=f(\eta)-\eta+1$。
	\end{enumerate}
\end{example}
	\begin{newproof}
		\begin{enumerate}
			\item
			令\[F\left( x \right) =f\left( x \right) -x\]
			则\[
				F\left( \frac{1}{2} \right) =f\left( \frac{1}{2} \right) -\frac{1}{2}=\frac{1}{2}>0\text{且}F\left( 1 \right) =f\left( 1 \right) -1=-1<0
				\]
			故\[
				F\left( \frac{1}{2} \right) F\left( 1 \right) <0
				\]
			又\[
				F\left( x \right) \text{在}\left[ \frac{1}{2},1 \right] \text{上连续}
				\]
			故由零点存在定理知	\[
				\exists \xi \in \left( \frac{1}{2},1 \right) 
				\]
			使得\[
				F\left( \xi \right) =0
				\]
			即\[
				f\left( \xi \right) =\xi 
				\]
			\item	
			令\[F\left( x \right) =e^{-x}\left[ f\left( x \right) -x \right] \]
			因\[F\left( 0 \right) =F\left( \xi \right) =0\text{且}F\left( x \right) \text{在}\left[ 0,\xi \right] \text{上连续,}\left( 0,\xi \right) \text{上可导}\]	
			由罗尔定理\[\exists \eta \in \left( 0,\xi \right) \]
			使得\[F'\left( \eta \right) =0\]
			即\[e^{-\eta}\left[ f'\left( \eta \right) -1-f\left( \eta \right) +\eta \right] =0\]
			又\[e^{-n}>0\]
			故\[f'\left( \eta \right) =f\left( \eta \right) -\eta +1\]
			\begin{note}
				\begin{align*}
					f^{''}\left( n \right) =f\left( n \right) -n+1
					\Longrightarrow {}&
					f'\left( x \right) -1=f\left( x \right) -x\\
					\Longrightarrow {}&
					\frac{f'\left( x \right) -1}{f\left( x \right) -x}=1\\
					\Longrightarrow {}&
					\int{\frac{f'\left( x \right) -1}{f\left( x \right) -x}dx=\int{dx}}\\
					\Longrightarrow {}&
					\ln \left| f\left( x \right) -x \right|=x+\ln \left| c \right|\\
					\Longrightarrow {}&
					e^{-x}\left( f\left( x \right) -x \right) =c
				\end{align*}
				故构造\[F\left( x \right) =e^{-x}\left[ f\left( x \right) -x \right] \]
			\end{note}	
				
		\end{enumerate}
	\end{newproof}

%14
\begin{example}
	设$f(x)$,$g(x)$均在$[1,6]$上连续,在$(1,6)$内可导,且$f(1)=5$,$f(5)=1$,$f(6)=12$,求证$\exists\xi\in(1,6)$,使得$f'(\xi)+g'(\xi)[f(\xi)-2\xi]=2$。
\end{example}
	\begin{newproof}
		令\[F\left( x \right) =e^{g\left( x \right)}\left[ f\left( x \right) -2x \right] \]
		因\[F\left( 1 \right) =3e^{g\left( 1 \right)}>0,F\left( 5 \right) =-9e^{g\left( 5 \right)}<0\]
		即\[F\left( 1 \right) F\left( 5 \right) <0\]
		又\[F\left( x \right) \text{在}\left[ 1,5 \right] \text{连续}\]
		由零点存在定理\[\exists x_0\in \left( 1,5 \right) \]
		使\[F\left( x_0 \right) =0\]
		另一方面\[F\left( 6 \right) =0=F\left( x_0 \right) \text{且}F\left( x \right) \text{在}\left[ x_0,6 \right] \text{上连续,}\left( x_0,6 \right) \text{可导}\]
		由罗尔定理\[\exists \xi \in \left( x_0,6 \right) \]
		使得\[F'\left( \xi \right) =0\]
		即\[e^{g\left( \xi \right)}\left[ f'\left( \xi \right) -2+e^{g\left( \xi \right)}\left( f\left( \xi \right) -2\xi \right) \right] =0\]
		又\[e^{g\left( \xi \right)}>0\]
		故\[f'\left( \xi \right) +e^{g\left( \xi \right)}\left[ f\left( \xi \right) -2\xi \right] =2\]
	\end{newproof}
	\begin{note}
		\begin{align*}
			f'\left( \xi \right) +g'\left( \xi \right) \left[ f\left( \xi \right) -2\xi \right] =2
			\Longrightarrow {}&
			\left[ f'\left( x \right) -2 \right] +g'\left( x \right) \left[ f\left( x \right) -2x \right] =0\\
			\Longrightarrow {}&
			\frac{f'\left( x \right) -2}{f\left( x \right) -2x}=-g'\left( x \right) \\
			\Longrightarrow {}&
			\int{\frac{f'\left( x \right) -2}{f\left( x \right) -2x}dx=-\int{g'\left( x \right) dx}}\\
			\Longrightarrow {}&
			\ln \left| f\left( x \right) -2x \right|=-g\left( x \right) +\ln \left| c \right|\\
			\Longrightarrow {}&
			e^{g\left( x \right)}\left[ f\left( x \right) -2x \right] =C
		\end{align*}
		故构造\[F(x) = e^{g\left( x \right)}\left[ f\left( x \right) -2x \right] \]
	\end{note}

%15
\begin{example}
	设$f(x)$在$[0,1]$上连续,在$(0,1)$内可导,$f(0)=0$,$分f(x)$在$(0,1)$内非零,试证明在$(0,1)$内至少存在一点$\xi$,使得$\frac{f'(\xi)}{f(\xi)}=\frac{f'(1-\xi)}{f(1-\xi)}$。
\end{example}
	\begin{newproof}
		令\[F\left( x \right) =f\left( x \right) f\left( 1-x \right) \ x\in \left[ 0,1 \right] \]
		因\[F\left( 0 \right) =F\left( 1 \right) =0\text{且}F\left( x \right) \text{在}\left[ 0,1 \right] \text{连续,}\left( 0,1 \right) \text{可导}\]
		由罗尔定理\[\exists \xi \in \left( 0,1 \right) \]
		使得\[F'\left( \xi \right) =0\]
		即\[f'\left( \xi \right) f\left( 1-\xi \right) -f\left( \xi \right) f'\left( 1-\xi \right) =0\]
		又\[\forall x\in \left( 0,1 \right) ,\text{有}f\left( x \right) \ne 0\]
		故\[\frac{f'\left( \xi \right)}{f\left( \xi \right)}=\frac{f'\left( 1-\xi \right)}{f\left( 1-\xi \right)}\]
	\end{newproof}
	\begin{note}
		\begin{align*}
			\frac{f'\left( \xi \right)}{f\left( \xi \right)}=\frac{f'\left( 1-\xi \right)}{f\left( 1-\xi \right)}
			\Longrightarrow {}&
			\frac{f'\left( x \right)}{f\left( x \right)}=\frac{f'\left( 1-x \right)}{f\left( 1-x \right)}\\
			\Longrightarrow {}&
			\int{\frac{f'\left( x \right)}{f\left( x \right)}}dx=\int{\frac{f'\left( 1-x \right)}{f\left( 1-x \right)}}dx\\
			\Longrightarrow {}&
			\ln \left| f\left( x \right) \right|=-\ln \left| f\left( 1-x \right) \right|+\ln \left| C \right|\\
			\Longrightarrow {}&
			f\left( x \right) f\left( 1-x \right) =C
		\end{align*}
		故构造\[F(x) = f\left( x \right) f\left( 1-x \right)\]
	\end{note}

%16
\begin{example}
	设$f(x)$在$[0,1]$上连续,在$(0,1)$内可导,$f(0)=0$,$f(x)$在$(0,1)$内非零,试证在$(0,1)$内至少存在一点$\xi$,使得$\frac{mf'(\xi)}{f(\xi)}=\frac{nf'(1-\xi)}{f(1-\xi)}$。
\end{example}
	\begin{newproof}
		令\[F\left( x \right) =\left[ f\left( x \right) \right] ^m\left[ f\left( 1-x \right) \right] ^n\]
		因\[F\left( 0 \right) =F\left( 1 \right) =0\text{且}F\left( x \right) \text{在}\left[ 0,1 \right] \text{连续,}\left( 0,1 \right) \text{可导}\]
		由罗尔定理\[\exists \xi \in \left( 0,1 \right) \]
		使得\[F'\left( \xi \right) =0\]
		即\[m\left[ f\left( \xi \right) \right] ^{m-1}f'\left( \xi \right) \left[ f\left( 1-\xi \right) \right] ^n-n\left[ f\left( 1-\xi \right) \right] ^{n-1}f'\left( 1-\xi \right) \left[ f\left( x \right) \right] ^m=0\]
		又\[\forall x\in \left( 0,1 \right) ,\text{有}f\left( x \right) \ne 0\]
		故\[\frac{mf'\left( \xi \right)}{f\left( \xi \right)}=\frac{nf'\left( 1-\xi \right)}{f'\left( 1-\xi \right)}\]
	\end{newproof}

%17
\begin{example}
	设$f(x)$在$[0,4]$上可导,且$f(0)=0$,$f(1)=1$,$f(4)=2$,证明至少存在一点$\xi\in(0,4)$,使得$f''(\xi)=-\frac{1}{3}$。
\end{example}
	\begin{newproof}
		令\[F\left( x \right) =f\left( x \right) +\frac{x^2}{6}-\frac{7}{6}x\]
		有\[F\left( 0 \right) =F\left( 1 \right) =F\left( 4 \right) =0\]
		又\[F\left( x \right) \text{在}\left[ 0,1 \right] \text{连续,}\left( 0,1 \right) \text{可导,}\left[ 1,4 \right] \text{连续,}\left( 1,4 \right) \text{可导}\]
		故由罗尔定理\[\exists \xi _1\in \left( 0,1 \right) \text{,}\xi _2\in \left( 1,4 \right) \]
		使\[F'\left( \xi _1 \right) =F'\left( \xi _2 \right) =0\]
		又\[F'\left( x \right) \text{在}\left[ \xi _1,\xi _2 \right] \text{内连续,}\left( \xi _1,\xi _2 \right) \text{可导}\]
		由罗尔定理\[\exists \xi \in \left( \xi _1,\xi _2 \right) \text{,使}F''\left( \xi \right) =0\]
		即\[f''\left( \xi \right) =-\frac{1}{3}\]
		得证
	\end{newproof}
	\begin{note}
		\begin{align*}
			f''\left( \xi \right) =-\frac{1}{3}
			\Longrightarrow {}&
			f''\left( x \right) =-\frac{1}{3}\\
			\Longrightarrow {}&
			f'\left( x \right) =-\frac{x}{3}+C_1\\
			\Longrightarrow {}&
			f\left( x \right) =-\frac{x^2}{6}+C_1x+C_2
		\end{align*}
		只需要找一个$g(x)$,使得$F\left( 0 \right) =F\left( 0 \right) =F\left( 1 \right) =F\left( 4 \right) $,即可得出结论
		\[\begin{cases}
			g\left( 0 \right) =f\left( 0 \right) =0\\
			g\left( 1 \right) =f\left( 1 \right) =1\\
			g\left( 4 \right) =f\left( 4 \right) =2\\
		\end{cases}\Longrightarrow \begin{cases}
			C_2=0\\
			\\
			-\frac{1}{6}+C_1+C_2=1\\
			\\
			-\frac{3}{8}+4C_1+C_2=2\\
		\end{cases}\Longrightarrow \begin{cases}
			C_1=\frac{7}{6}\\
			C_2=0\\
		\end{cases}\]

	\end{note}

%18
\begin{example}
	设奇函数$f(x)$在$[-1,1]$上具有二阶导数,且$f(1)=1$,证明
	\begin{enumerate}
		\item 存在一点$\xi\in(0,1)$,使得$f'(\xi)=1$;
		\item 存在一点$\eta\in(-1,1)$,使得$f''(\eta)+f'(\eta)=1$。
	\end{enumerate}
\end{example}
	\begin{newproof}
		\begin{enumerate}
			\item 
			令\[F\left( x \right) =f\left( x \right) -x\]
		由题\[f\left( x \right) 为\text{奇函数}\Longrightarrow f\left( -1 \right) =-f\left( 1 \right) =-1\text{且}f\left( 0 \right) =0\]
		因\[F\left( 0 \right) =F\left( 1 \right) =0\text{且}F\left( x \right) \text{在}\left[ 0,1 \right] \text{连续,}\left( 0,1 \right) \text{可导}\]
		由罗尔定理\[\exists \xi \in \left( 0,1 \right) \]
		使\[F'\left( \xi \right) =0\]
		即\[f'\left( \xi \right) =1\]
			\item
			令\[G\left( x \right) =e^x\left[ f'\left( x \right) -1 \right] =e^xF'\left( x \right) \]
			由$F(x)$为奇函数得\[F\left( -1 \right) =F\left( 0 \right) =0\]
			由罗尔定理\[\exists \xi \in \left( 0,1 \right) \]
			使得\[F'\left( -\xi \right) =0\]
			则\[G\left( \xi \right) =G\left( -\xi \right) =0\text{且}G\left( x \right) \text{在}\left[ -\xi \text{,}\xi \right] \text{上连续,}\left( -\xi ,\xi \right) \text{可导}\]
			由罗尔定理\[\exists \eta \in \left( -\xi ,\xi \right) \subset \left( -1,1 \right) \]
			使得\[G'\left( \eta \right) =0\]
			即\[f''\left( \eta \right) +f'\left( \eta \right) =1\]

		\end{enumerate}
		
	\end{newproof}

%19
\begin{example}
	设$f(x)$在$[a,b]$上连续,在$(a,b)$内可导,且$f(a)f(b)>0$,$f(a)f(\frac{a+b}{2})<0$,证明对任何实数$k$,必定存在$\xi\in(a,b)$,使得$f'(\xi)=kf(\xi)$。
\end{example}
	\begin{newproof}
		令\[F\left( x \right) =e^{-kx}f\left( x \right) \,\,     x\in \left[ a,b \right] \]
		不妨设\[f\left( a \right) >0\]
		则由题意知\[f\left( b \right) >0,f\left( \frac{a+b}{2} \right) <0\]
		即\[f\left( a \right) f\left( \frac{a+b}{2} \right) <0,   f\left( \frac{a+b}{2} \right) f\left( b \right) <0\]
		因\[f\left( x \right) \text{在}\left[ a,\frac{a+b}{2} \right] ,\left[ \frac{a+b}{2},b \right] \text{上连续}\]
		由零点存在定理\[\exists x_1\in \left( a,\frac{a+b}{2} \right) ,x_2\in \left( \frac{a+b}{2},b \right) \]
		使得\[f\left( x_1 \right) =f\left( x_2 \right) =0\]
		即\[F\left( x_1 \right) =F\left( x_2 \right) =0\]
		又因\[F\left( x \right) \text{在}\left[ x_1,x_2 \right] \text{上连续,}\left( x_1,x_2 \right) \text{上可导}\]
		由罗尔定理\[\exists \xi \in \left( x_1,x_2 \right) \]
		使得\[F'\left( \xi \right) =0\]
		即\[e^{-k\xi}\left[ f'\left( \xi \right) -kf\left( \xi \right) \right] =0\]
		又\[e^{-k\xi}>0\]
		故\[f'\left( \xi \right) =kf\left( \xi \right) \]
		得证
	\end{newproof}

%20
\begin{example}
	设$f(\xi)$在闭区间$[a,b]$上连续,在$(a,b)$内可导,且$f(a)=f(b)=\frac{a}{2}$,$f(\frac{a+b}{2})=a+b$,其中$0<a<b$,证明对任意的$\lambda$,存在$\xi\in(a,b)$,使得\[f'(\xi)=\lambda\left[f(\xi)-\frac{1}{2}\xi\right]+\frac{1}{2}\]
\end{example}
	\begin{newproof}
		令\[F\left( x \right) =e^{-\lambda x}\left[ f\left( x \right) -\frac{x}{2} \right] \,\,   g\left( x \right) =f\left( x \right) -\frac{x}{2}\]
		由题意得\[g\left( a \right) =f\left( a \right) -\frac{a}{2}=0\text{,}g\left( b \right) =f\left( b \right) -\frac{b}{2}=\frac{a-b}{2}<0\text{,}g\left( \frac{a+b}{2} \right) =f\left( \frac{a+b}{2} \right) -\frac{a+b}{4}>0\]
		得\[g\left( b \right) g\left( \frac{a+b}{2} \right) <0\]
		因\[g\left( x \right) \text{在}\left[ \frac{a+b}{2},b \right] \text{上连续}\]
		由零点存在定理\[\exists x_1\in \left( \frac{a+b}{2},b \right) \]
		使得\[g\left( x_1 \right) =0\]
		故\[F\left( a \right) =F\left( x_1 \right) =0\]
		又\[F\left( x \right) \text{在}\left[ a,x_1 \right] \text{连续}\left( a,x_1 \right) \text{可导}\]
		由罗尔定理\[\exists \xi \in \left( a,x_1 \right) \]
		使\[F'\left( \xi \right) =0\]
		即\[e^{-\lambda \xi}\left[ f^{''}\left( \xi \right) -\frac{1}{2}-\lambda \left( f'\left( \xi \right) -\frac{\xi}{2} \right) \right] =0\]
		即\[f'\left( \xi \right) =\lambda \left[ f\left( \xi \right) -\frac{\xi}{2} \right] +\frac{1}{2}\]

	\end{newproof}
%21
\begin{example}
	设$f(x)$在$[0,1]$上具有二阶导数,且$f'(0)=0$,试证明存在$\xi\in(0,1)$,使$f''(\xi)=\frac{2f'(\xi)}{1-\xi}$。
\end{example}
	\begin{newproof}
		令\[F\left( x \right) =\left( 1-x \right) ^2f'\left( x \right) \,\,   x\in \left[ 0,1 \right] \]
		因\[F\left( 0 \right) =F\left( 1 \right) =0\text{且}F\left( x \right) \text{在}\left[ 0,1 \right] \text{连续,}\left( 0,1 \right) \text{可导}\]
		由罗尔定理\[\exists \xi \in \left( 0,1 \right) \]
		使\[F'\left( \xi \right) =0\]
		即\[-2\left( 1-\xi \right) f'\left( \xi \right) +\left( 1-\xi \right) ^2f^{''}\left( \xi \right) =0\]
		又\[1-\xi \ne 0\]
		故\[f^{''}\left( \xi \right) =\frac{2f'\left( \xi \right)}{1-\xi}\]
		得证

	\end{newproof}

%22
\begin{example}
	设$f(x)$在$[0,1]$上具有二阶导数,且$f(0)=f(1)=0$,试证明存在$\xi\in(0,1)$,使$f''(\xi)=\frac{2f'(\xi)}{1-\xi}$。
\end{example}
	\begin{newproof}
		令\[F\left( x \right) =\left( 1-x \right) ^2f'\left( x \right) \]
		因\[f\left( 0 \right) =f\left( 1 \right) =0\text{且}f\left( x \right) \text{在}\left[ 0,1 \right] \text{连续,}\left( 0,1 \right) \text{可导}\]
		由罗尔定理\[\exists \xi \in \left( \xi ,1 \right) \]
		使\[F'\left( \xi \right) =0\]
		即\[-2\left( 1-\xi \right) f'\left( \xi \right) +\left( 1-\xi \right) ^2f^{''}\left( \xi \right) =0\]
		又\[1-\xi \ne 0\]
		故\[f^{''}\left( \xi \right) =\frac{2f'\left( \xi \right)}{1-\xi}\]
		得证
	\end{newproof}

%23
\begin{example}
	设$f(x)$在$[a,b]$上具有二阶导数,$f(0)=f(1)=f'(0)=f'(1)$,证明存在$\xi\in(0,1)$,使得$f''(\xi)=f(\xi)$。
\end{example}
	\begin{newproof}
		令\[F\left( x \right) =e^x\left[ f'\left( x \right) -f\left( x \right) \right] \]
		因\[F\left( 0 \right) =F\left( 1 \right) =0\text{且}F\left( x \right) \text{在}\left[ 0,1 \right] \text{连续,}\left( 0,1 \right) \text{可导}\]
		由罗尔定理\[\exists \xi \in \left( 0\text{,}1 \right) \]
		使\[F'\left( \xi \right) =0\]
		即\[e^{\xi}\left[ f^{''}\left( \xi \right) -f'\left( \xi \right) +f'\left( \xi \right) -f\left( \xi \right) \right] =0\]
		又\[e^{\xi}>0\]
		即\[f^{''}\left( \xi \right) =f\left( \xi \right) \]
		得证
	\end{newproof}

%24
\begin{example}
	设$f(x)$,$g(x)$在$[a,b]$上连续,在$(a,b)$内具有二阶导数,且存在相等的最大值,又$f(a)=g(a)$,$f(b)=g(b)$,证明存在$\xi\in(a,b)$,使得$f''(\xi)=g''(\xi)$。
\end{example}
	\begin{newproof}
		令\[F\left( x \right) =f\left( x \right) -g\left( x \right) , x\in \left[ a,b \right] \]
		若$f\left( x \right) ,g\left( x \right) $在同一点$x_0$处取得最大值,则\[F\left( a \right) =F\left( b \right) =F\left( x_0 \right) =0\]
		由罗尔定理,结论是显然的。\\
		若$f(x),g(x)$不在同一点处取到最大值,不妨设$f(x)$在$x_1$处取最大值, $g(x)$在$x_2$处取最大值,且$x_1<x_2$\\
		则有\[F\left( x_1 \right) =f\left( x_1 \right) -g\left( x_1 \right) \geqslant 0\text{,}F\left( x_2 \right) =f\left( x_2 \right) -g\left( x_2 \right) \leqslant 0\]
		因为$F(x)$在$\left[ x_1,x_2 \right] $上连续
		由零点存在性定理得\[\exists x_0\in \left( x_1,x_2 \right) \]
		使得\[F\left( x_0 \right) =0\]
		故\[F\left( a \right) =F\left( b \right) =F\left( x_0 \right) =0\]
		连续使用罗尔定理,结论成立
	\end{newproof}

%25
\begin{example}
	设$a_1$,$a_2$,$\cdots$,$a_n$为$n$个实数,并满足$a_1-\frac{a_2}{3}+\cdots+(-1)^n\frac{a_n}{2n-1}=0$,证明方程$a_1\cos x+a_2\cos 3x+\cdots+a_n\cos (2n-1)x=0$在$(0,\frac{\pi}{2})$至少有一实根。
\end{example}
	\begin{newproof}
		令\[F\left( x \right) =a_1\sin x+\frac{a_2}{3}\sin 3x+\cdot \cdot \cdot \cdot +\frac{a_n}{2n-1}\sin \left( 2n-1 \right) x\]
		有\[F\left( 0 \right) =0,F\left( \frac{\pi}{2} \right) =a_1-\frac{a_2}{3}+\cdot \cdot \cdot +\frac{a_n}{2n-1}\left( -1 \right) ^{n-1}=0\]
		即\[F\left( 0 \right) =F\left( \frac{\pi}{2} \right) =0\]
		又\[F\left( x \right) \text{在}\left[ o,\frac{\pi}{2} \right] \text{上连续,}\left( o,\frac{\pi}{2} \right) \text{可导}\]
		由罗尔定理\[\exists \xi \in \left( o,\frac{\pi}{2} \right) \]
		使得\[F'\left( \xi \right) =0\]
		且\[F'\left( x \right) =a_1\cos x+a_2\cos 3x+\cdot \cdot +a_n\cos \left( 2n-1 \right) x\]
		结论成立
	\end{newproof}

%26	
\begin{example}
	设$\frac{a_0}{n+1}+\frac{a_1}{n}+\cdots+\frac{a_{n-1}}{2}+a_n=0$,证明方程$a_0x^n+a_1x^{n-1}+\cdots+a_n=0$至少有一个不小于$1$的实根。
\end{example}
	\begin{newproof}
		令\[F\left( x \right) =f\left( x \right) -g\left( x \right) \,\,   x\in \left[ a,b \right] \]
		若$f\left( x \right) ,g\left( x \right) $在同一点$x_0$处取得最大值\\
		则\[F\left( a \right) =F\left( b \right) =F\left( x \right) =0\]
		由罗尔定理,结论是显然的.\\
		若$f\left( x \right) ,g\left( x \right) $不在同一点取得最大值,不妨设$f\left( x \right) \text{在}x=x_1$处取最大值,$g\left( x \right) \text{在}x=x_2$处取最大值,且$x_1<x_2$,则有
		\[F\left( x_1 \right) =f\left( x_1 \right) -g\left( x_1 \right) \geqslant 0\text{,}F\left( x_2 \right) =f\left( x_2 \right) -g\left( x_2 \right) \leqslant 0\]
		因$F\left( x \right) \text{在}\left[ x_1,x_2 \right] $\\\
		由零点存在定理\[\exists x_0\in \left( x_1,x_2 \right) \]
		使得\[F\left( x_0 \right) =0\]
		故\[F\left( a \right) =F\left( b \right) =F\left( x_0 \right) =0\]
		连续使用罗尔定理,结论成立。
	\end{newproof}

%27
\begin{example}
	证明方程$2^x-x^2=1$有且仅有三个实根。
\end{example}
	\begin{newproof}
		令\[F\left( x \right) =2^x-x^2-1\]
		有\[F\left( 0 \right) =0,F\left( 1 \right) =0,F\left( 4 \right) =-1,F\left( 5 \right) =6>0\]
		因\[F\left( 4 \right) \cdot F\left( 5 \right) <0\]
		由零点存在定理,至少存在一点$\xi \in \left( 4,5 \right) $,使得\[F\left( \xi \right) =0\]
		故$F\left( x \right) \text{至少存在三}个\text{实根}$\\
		再证:$F\left( x \right) $仅有三个实根\\
		若\[\exists x_1,x_2,x_3,x_4\]
		使得\[F\left( x_1 \right) =F\left( x_2 \right) =F\left( x_3 \right) =F\left( x_4 \right) =0\]
		其中$x_1,x_2,x_3,x_4$互不相等\\
		由罗尔定理\[F'\left( x_{1}^{'} \right) =F'\left( x_{2}^{'} \right) =F'\left( x_{3}^{'} \right) =0 \Rightarrow F^{''}\left( x_{1}^{''} \right) =F^{''}\left( x_{2}^{''} \right) =0\Rightarrow F^{'''}\left( x_{0}^{'''} \right) =0\]
		事实上\[F'\left( x \right) =\ln 2\cdot 2^x-2x\]
		\[F^{''}\left( x \right) =\left( \ln 2 \right) ^2\cdot 2^x-2\]
		\[F^{'''}\left( x \right) =\left( \ln 2 \right) ^3\cdot 2^x\ne 0\]
		与$F^{'''}\left( x_{0}^{'''} \right) =0$矛盾,故$F(x)$仅三个零点。
	\end{newproof}

%28
\begin{example}
	若$f(x)$为可导函数,$g(x)$为连续函数,试证明在$f(x)$的两个零点之间,一定有$f'(x)-kf(x)g(x)=0$的零点。
\end{example}
	\begin{newproof}
		设\[f\left( a \right) =f\left( b \right) =0\ \text{其中}a<b\]
		令\[F\left( x \right) =f\left( x \right) e^{-k\int_a^x{g\left( t \right) dt}}\]
		则有\[F\left( a \right) =F\left( b \right) =0\]
		又\[F\left( x \right) \text{在}\left[ a,b \right] \text{连续},\left( a,b \right) \text{可导}\]
		则由罗尔定理\[\exists \xi \in \left( a,b \right) \]
		使得\[F'\left( \xi \right) =0\]
		因\[F'\left( x \right) =e^{-k\int_a^x{g\left( t \right) dt}}\left[ f'\left( x \right) -kg\left( x \right) f\left( x \right) \right] \]
		又\[e^{-k\int_a^{\xi}{g\left( t \right) dt}}>0\]
		故\[F'\left( \xi \right) =0\Rightarrow f'\left( \xi \right) -kg\left( \xi \right) f\left( \xi \right) =0\]
		命题得证
	\end{newproof}

%29
\begin{example}
	设$f(x)$在$[0,1]$上连续,在$(0,1)$内可导,且满足$f(1)=k\int_{a}^{b}xe^{1-x}f(x)\diff x(k>1)$,证明至少存在一点$\xi\in(0,1)$,使得$f'(\xi)=(1-\frac{1}{\xi})f(\xi)$。
\end{example}
	\begin{newproof}
		由积分中值定理\[\exists x_0\in \left( 0,\frac{1}{k} \right) \]
		使得\[\int_0^{\frac{1}{k}}{xe^{1-x}f\left( x \right) dx}=\frac{1}{k}x_0e^{1-x_0}f\left( x_0 \right) \]
		令\[F\left( x \right) =xe^{1-x}f\left( x \right) \]
		则\[F\left( 1 \right) =f\left( 1 \right) =k\cdot \frac{1}{k}x_0e^{1-x_0}f\left( x_0 \right) =F\left( x_0 \right) \]
		又\[F\left( x \right) \text{在}\left[ x_0,1 \right] \text{连续}\]
		使得\[F'\left( \xi \right) =0\]
		由\[e^{1-\xi}>0,F'\left( \xi \right) =0\]
		得\[\xi f'\left( \xi \right) =\left( \xi -1 \right) f\left( \xi \right) \]
		又\[\xi \ne 0\]
		故\[f'\left( \xi \right) =\left( \xi -1 \right) f\left( \xi \right) \]
		得证
	\end{newproof}

%30
\begin{example}
	设$f(x)$,$g(x)$在$[a,b]$上连续,试证明存在$\xi\in(a,b)$,使得\[f(\xi)\int_{\xi}^{b}g(x)\diff x=g(\xi)\int_{a}^{\xi}f(x)\diff x\]
\end{example}
	\begin{newproof}
		令\[F\left( x \right) =\int_a^x{f\left( t \right) dt}\int_x^b{g\left( t \right) dt}\]
		由\[F\left( a \right) =F\left( b \right) =0\]
		又\[F\left( x \right) \text{在}\left[ a,b \right] \text{连续,}\left( a,b \right) \text{内可导}\]
		故由罗尔定理\[\exists \xi \in \left( a,b \right) \]
		使\[F'\left( \xi \right) =0\]
		即\[f\left( \xi \right) \int_{\xi}^b{g\left( x \right) dx}=g\left( \xi \right) \int_a^{\xi}{f\left( x \right) dx}\]
	\end{newproof}
	\begin{note}
		\begin{align*}
			f\left( \xi \right) \int_{\xi}^b{g\left( x \right) dx}=g\left( \xi \right) \int_a^{\xi}{f\left( x \right) dx}
			\Longrightarrow {}&
			\frac{\,\,f\left( \xi \right)}{\int_a^{\xi}{f\left( x \right) dx}}=\frac{g\left( \xi \right)}{\int_{\xi}^b{g\left( x \right) dx}}\\
			\Longrightarrow {}&
			\int{\frac{\,\,f\left( x \right)}{\int_a^x{f\left( t \right) dt}}dx}=\int{\frac{g\left( x \right)}{\int_x^b{g\left( t \right) dt}}dx}\\
			\Longrightarrow {}&
			ln|\int_a^x{f\left( t \right) dt}|=-\ln |\int_x^b{g\left( t \right) dt}|\\
			\Longrightarrow {}&
			\int_a^x{f\left( t \right) dt}\int_x^b{g\left( t \right) dt}=C
		\end{align*}
	\end{note}
\begin{example}
	已知函数$f(x)$在$[0,1]$上连续,在$(0,1)$内二阶可导,且$f(1)=0$,设函数$g(x)=x^2f(x)$,证明至少存在一点$\xi\in(0,1)$,使得$g''(\xi)=0$。
\end{example}

%31
\begin{example}
	设$f(x)$在$[0,1]$上二阶可导,且$f(1)>0$,$\lim_{x\to 0^+}\frac{f(x)}{x}<0$,证明
	\begin{enumerate}
		\item 方程$f(x)=0$在区间$(0,1)$内至少存在一个实根;
		\item 方程$f(x)f''(x)+[f'(x)]^2=0$在区间$(0,1)$内至少存在两个不同的实根。
	\end{enumerate}
\end{example}
	\begin{newproof}
		因\[g\left( 0 \right) =g\left( 1 \right) =0\text{且}g\left( x \right) \text{在}\left[ 0,1 \right] \text{连续,}\left( 0,1 \right) \text{可导}\]
		则由罗尔定理\[\exists x_1\in \left( 0,1 \right) \]
		使得\[g'\left( x_1 \right) =0\]
		因\[g'\left( x \right) =2xf\left( x \right) +x^2f'\left( x \right) \]
		故\[g'\left( 0 \right) =g'\left( x_1 \right) =0\]
		又\[g'\left( x \right) \text{在}\left[ 0,x_1 \right] \text{连续,}\left( 0,x_1 \right) \text{可导}\]
		则由罗尔定理\[\exists \xi \in \left( 0,x \right) \]
		使得\[g''\left( \xi \right) =0\]
		得证
	\end{newproof}

%32
\begin{example}
	已知函数$f(x)$在$[a,b]$上连续,在$(a,b)$内二阶可导,且$f(a)=f(b)$,$f'(x)\neq 0$,证明$\exists\xi\in(a,b)$,使得$[f'(\xi)]^2=f(\xi)f''(\xi)$。
\end{example}
	\begin{newproof}
		\begin{enumerate}
			\item \[\lim_{x\rightarrow 0^+} \frac{f\left( x \right)}{x}<0\Longleftrightarrow \exists \delta >0,\text{当}0<x<\delta \text{时,有}\frac{f\left( x \right)}{x}<0\]
			故\[\exists x_0\in \left( 0,\delta \right) \]
			使\[\frac{f\left( x_0 \right)}{x_0}<0\Longleftrightarrow f\left( x_0 \right) <0\]
			又\[f\left( x_0 \right) <0,f\left( 1 \right) >0,f\left( x \right) \text{在}\left[ x,1 \right] \text{连续}\]
			由零点存在定理\[\exists \xi \in \left( 0,1 \right) \]
			使\[f\left( \xi \right) =0\]
			\item 令\[F\left( x \right) =f\left( x \right) f'\left( x \right) \]
			因$\lim_{x\rightarrow 0^+} \frac{f\left( x \right)}{x}\text{存在,且}f\left( x \right) \text{在}x=0\text{处连续}$\\
			故\[f\left( 0 \right) =\lim_{x\rightarrow 0^+} f\left( x \right) =0\]
			由拉格朗日中值定理\[\exists x_1\in \left( x_0,1 \right) \]
			使\[f'\left( x_1 \right) =\frac{f\left( 1 \right) -f\left( x \right)}{1-x_0}>0\]
			又\[\lim_{x\rightarrow 0^+} \frac{f\left( x \right)}{x}=\lim_{x\rightarrow 0^+} \frac{f\left( x \right) -f\left( 0 \right)}{x-0}=f'\left( 0 \right) <0 \text{且}f'\left( x \right) \text{在}\left[ 0,x_1 \right] \text{上连续}\]
			由零点存在定理\[\exists x_2\in \left( 0,x_1 \right) \]
			使\[f'\left( x_1 \right) =\frac{f\left( 1 \right) -f\left( x \right)}{1-x_0}>0\]
			使\[f'\left( x_2 \right) =0\]
			故\[F\left( 0 \right) =F\left( x_2 \right) =F\left( \xi \right) =0\]
			由罗尔定理\[\exists x',x''\text{且}x'\ne x''\]
			使\[F'\left( x' \right) =F'\left( x'' \right) =0\]
			得证
		\end{enumerate}
	\end{newproof}
	
%34
\begin{example}
	$f(x)$在$[a,b]$上连续,在$(a,b)$内二阶可导,且$f(a)=f(b)=\int_{a}^{b}f(x)\diff x=0$,证明
	\begin{enumerate}
		\item $\exists\xi_1\in(a,b)$,使得$f''(\xi_1)=f(\xi_1)$;
		\item $\exists\xi_2\in(a,b)$,使得$f''(\xi_2)-3f'(\xi_2)+2f(\xi_2)=0$。
	\end{enumerate}
\end{example}

\begin{example}
	设$f(x)$在$[0,\pi]$上连续,在$(0,\pi)$内二阶可导,且$f(0)+f(\pi)=0$,证明$\exists\xi\in(0,\pi)$,使得$f''(\xi)+f(\xi)=0$。
\end{example}

\begin{example}
	设$f(x)$在$[0,+\infty)$上有连续导数,$f(0)=1$,且对一切$x\geqslant0$有$|f(x)|\leqslant e^{-x}$,证明存在一点$\xi\in(0,+\infty)$,使得$f'(\xi)=-e^{-\xi}$。
\end{example}

\begin{example}
	设$f(x)$在$[0,1]$上二阶可导,且$f(0)=f(1)=0$,证明至少存在一点$\xi\in(0,1)$,使得$\xi^2f''(\xi)+4\xi f(\xi)=2f(\xi)=0$。
\end{example}

\begin{example}
	设函数$f(x)$具有二阶导数,且$f(0)=0$,证明存在$\xi\in\left(-\frac{\pi}{2},\frac{\pi}{2}\right)$,使得\[f''(\xi)=f(\xi)(1+\tan^2\xi)\]
\end{example}

\section{k值法}

\begin{example}
	设$f(x)$在$[a,b]$上二阶可导,$f(a)=f(b)=0$,证明对每个$x\in(a,b)$,存在$\xi\in(a,b)$,使得$f(x)=\frac{f''(\xi)}{2}(x-a)(x-b)$。
\end{example}

\begin{example}
	已知函数$f(x)$在$[0,1]$上三阶可导,且$f(0)=-1$,$f(1)=0$,$f'(0)=0$,证明存在一点$\xi\in(0,1)$,使得$f(x)=-1+x^2+\frac{x^2(x-1)}{3!}f'''(\xi)x\in(0,1)$。
\end{example}

\begin{example}
	设$f(x)$在$(0,1)$内有三阶导数,$0<a<b<1$,证明存在$\xi\in(a,b)$,使得$f(b)=f(a)+\frac{1}{2}(b-a)[f'(a)+f'(b)]-\frac{(b-a)^3}{12}f'''(\xi)$。
\end{example}

\begin{example}
	设$f(x)$在$a\leqslant x\leqslant b$上连续,在$(a,b)$内二阶可导,证明在$a<x<b$上有\[\frac{\frac{f(x)-f(a)}{x-a}-\frac{f(b)-f(a)}{b-a}}{x-b}=\frac{1}{2}f''(\xi)\quad(a<\xi<b)\]
\end{example}

\begin{example}
	设$f(x)$在$[a,b]$上具有连续的二阶导数,求证$\exists\in(a,b)$,使得\[\int_{a}^{b}f(x)\diff x=(b-a)f(\frac{a+b}{2})+\frac{1}{24}(b-a)^3f'''(\xi)\]
\end{example}

\begin{example}
	设$f(x)$在$[a,b]$上二阶可导,求证至少存在一点$\xi\in(a,b)$,使得\[\int_{a}^{b}f(x)\diff x=(b-a)\frac{f(a)+f(b)}{2}-\frac{1}{12}f''(\xi)(b-a)^3\]
\end{example}

\begin{example}
	设$f(x)$在$[a,b]$上连续,在$(a,b)$内二阶可导,求证$\exists\xi\in(a,b)$,使得\[f(b)-2f(\frac{a+b}{2})+f(a)=\frac{(b-a)^2}{4}f''(\xi)\]
\end{example}

\begin{example}
	设$f(x)$在$[a,b]$上可导,在$(a,b)$内二阶可导,若$a<c<b$,证明存在$\xi\in(a,b)$,使得$\frac{f(a)}{(a-b)(a-c)}+\frac{f(b)}{(b-a)(b-c)}+\frac{f(c)}{(c-a)(c-b)}=\frac{1}{2}f''(\xi)$。
\end{example}

\begin{example}
	设有实数$a_1$,$a_2$,$\cdots$,$a_n$,其中$a_1<a_2<\cdots<a_n$,函数$f(x)$在$[a_1,a_n]$上具有$n$阶导数,并满足$f(a_1)=f(a_2)=\cdots=f(a_n)=0$,证明对任意的$c\in[a_1,a_n]$,都相应的有$\xi\in(a_1,a_n)$,使得$f(c)=\frac{(c-a_1)(c-a_2)\cdots(c-a_n)}{n!}f^{(n)}(\xi)$。
\end{example}

\section{拉格朗日中值定理——弦线法}
\begin{example}
	设不恒为零的函数$f(x)$在$[a,b]$上连续,在$(a,b)$内可导,且$f(a)=f(b)$,证明在$(a,b)$内至少存在一点$\xi$,使得$f'(\xi)>0$。
\end{example}

\begin{example}
	设$f(x)$在$[0,1]$上连续,在$(0,1)$内可导,且$f(0)=0$,证明如果$f(x)$在$(0,1)$内不恒等于零,则必定存在一点$\xi\in(0,1)$,使得$f'(\xi)f(\xi)>0$。
\end{example}

\begin{example}
	设$f(x)$在$[a,b]$上连续,在$(a,b)$内具有二阶导数,且$f(a)=f(b)=0$,$f(c)<0(a<c<b)$,证明至少存在一点$\xi\in(a,b)$,使得$f''(\xi)>0$。
\end{example}

\begin{example}
	设$f(x)$在$[a,b]$上连续,在$(a,b)$内二阶可导,连接点$A(a,f(a))$,$B(b,f(b))$的直线段$AB$与曲线$y=f(x)$相交于点$C(c,f(c))(a<c<b)$,证明$\exists\in\xi(a,b)$使得$f''(\xi)=0$。
\end{example}

\begin{example}
	设$f(X)$在$[a,b]$上连续,在$(a,b)$内可导,又$f(x)$不是线性函数,且$f(b)>f(a)$,试证明$\exists\xi\in(a,b)$,使得$f'(\xi)>\frac{f(b)-f(a)}{b-a}$。
\end{example}

\begin{example}
	\color{red}设$f(X)$在$[0,1]$上可微,$f(0)=0$,$f(1)=1$,$k_1$,$k_2$,$\cdots$,$k_n$为$n$个正数,证明在$[0,1]$内存在一组互不相等的数$x_1$,$x_2$,$\cdots$,$x_n$使得$\sum_{i=1}^n\frac{k_i}{f'(x_i)}=\sum_{i=1}^nk_i$\color{black}。
\end{example}

\begin{example}
	\color{red}已知$f(x)$在$[0,1]$上连续,在$(0,1)$内可导,且$f(0)=0$,$f(1)=1$,证明
	\begin{enumerate}
		\item 存在$\xi\in(0,1)$,使得$f(\xi)=1-\xi$\color{black};
		\color{red}\item 存在两个不同的$\alpha$,$\beta\in(0,1)$,使得$f'(\alpha)f'(\beta)=1$\color{black}。
	\end{enumerate}
\end{example}

\begin{example}
	\color{red}设$f(x)$在$[0,1]$上连续,在$(0,1)$内可导,且$f(0)=0$,$f(1)=1$,试证明对于任意给定的正数$a$,$b$,在$(0,1)$内存在不同的点$\xi$和$\eta$使得$\frac{a}{f'(\xi)}+\frac{b}{f'(\eta)}=a+b$\color{black}。
\end{example}

\begin{example}
	设$f(x)$在$[a,b]$上可导,$f(0)=0$,$f(1)=1$,试证明在区间$[0,1]$上存在两个不同的点$x_1$,$x_2$使得$\frac{1}{f'(x_1)}+\frac{1}{f'(x_2)}=2$。
\end{example}

\begin{example}
	已知$f(x)$在$[0,1]$上连续,在$(0,1)$内可导,且$f(0)=0$,$f(1)=1$,证明存在互不相等的$\xi_1$,$\xi_2$,$\cdots$,$\xi_8\in(0,1)$,使得对正的常数$\ln H$,$\ln A^2$,$\ln E^2$,$\ln P^2$,$\ln Y^2$,$\ln R$,$\ln W$,$\ln N$满足下式\[e^{\frac{\ln H}{f'(\xi_1)}+\frac{\ln N}{f'(\xi_2)}+\frac{\ln W}{f'(\xi_3)}+\frac{\ln R}{f'(\xi_4)}+\frac{\ln A^2}{f'(\xi_5)}+\frac{\ln P^2}{f'(\xi_6)}+\frac{\ln Y^2}{f'(\xi_7)}+\frac{\ln E^2}{f'(\xi_8)}}=HAPPY\cdot NEW\cdot YEAR\]
\end{example}

\begin{example}
	设$f(x)$在$[0,1]$上连续,在$(0,1)$内可导,且$f(0)=0$,$f(1)=1$,$a$,$b$为给定的正数,证明$\exists\xi$,$\eta$,$0<\xi<\eta<1$,使得$af'(\xi)+bf'(\eta)=a+b$。
\end{example}

\section{拉格朗日中值定理——作为函数的表达}

\begin{example}
	设$f(x)$在$[0,+\infty)$上可微,且$f(0)=0$,并设有实数$A>0$,使得$|f'(x)|\leqslant A|f(x)|$,在$[0,+\infty)$成立,试证明在$[0,+\infty)$上$f(x)\equiv 0$。
\end{example}

\begin{example}
	设$[0,a]$上$|f''(x)|\leqslant M$,$f(x)$在$(0,a)$内取最大值,试证明\[|f'(0)|+|f'(a)|\leqslant Ma\]
\end{example}

\begin{example}
	证明若函数$f(x)$在$(0,+\infty)$内可微,且$\lim_{n\to+\infty}f'(x)=0$,则$\lim_{n\to+\infty}\frac{f(x)}{x}=0$。
\end{example}

\begin{example}
	设$f(x)$在$[a,+\infty)$上可导,且当$x>a$时,$f'(x)>k>0$,其中$k为常数$,证明若$f(a)<0$,则方程$f(x)=0$在$[a,a-\frac{f(a)}{k}]$内有且仅有一个实根。
\end{example}

\begin{example}
	设$f(x)$在有限区间$(a,b)$内可导,且$f'(x)$在该区间内有界,证明
	\begin{enumerate}
		\item $f(x)$在$(a,b)$内有界;
		\item $\lim_{x\to a^+}f(x)$与$\lim_{x\to b^-}f(x)$均存在。
	\end{enumerate}
\end{example}

\begin{example}
	证明若函数$f(x)$在开区间$(a,b)$内可导且无界,则$f'(x)$在$(a,b)$内也无界。
\end{example}

\begin{example}
	\begin{enumerate}
		\item 设$f(x)$在$(a,+\infty)$内可导,则$\lim_{x\to+\infty}f(x)$与$\lim_{x\to+\infty}f'(x)$都存在,证明$\lim_{x\to+\infty}f'(x)=0$;
		\item 设$f(x)$在$(-\infty,+\infty)$内可导,且$\lim_{x\to\infty}f(x)$与$\lim_{x\to\infty}f'(x)$都存在,证明$\lim_{x\to\infty}f'(x)=0$。
	\end{enumerate}
\end{example}

\begin{example}
	设$f(x)$在$(a,+\infty)$内可导,且$\lim_{x\to+\infty}f'(x)=A>0$,证明$\lim_{x\to+\infty}f(x)=+\infty$。
\end{example}

\begin{example}
	设$f(x)$在$[0,+\infty)$上可导,且当$x>0$时,$a<f'(x)<\frac{1}{x^2}$,证明$\lim_{x\to+\infty}f(x)$存在。
\end{example}

\begin{example}
	设$f(x)$在$[0,1]$上有连续导数,且$f(0)=f(1)=0$,证明$|\int_{a}^{b}f(x)\diff x|\leqslant\frac{M}{4}$,其中$M$是$|f'(x)|$在$[0,1]$上的最大值。
\end{example}

\begin{example}
	设$f(x)$在$[0,1]$上有连续导数,且$f(0)=0$,证明$|\int_{a}^{b}f(x)\diff x|\leqslant\frac{M}{2}$,其中$M$是$|f'(x)|$在$[0,1]$上的最大值。
\end{example}

\begin{example}
	设$f(x)$在$[0,1]$上有连续导数,且$f(a)=f(b)=0$,证明$|\int_{a}^{b}f(x)\diff x|\leqslant\frac{(b-a)^2M}{4}$,其中$M$是$|f'(x)|$在$[0,1]$上的最大值。
\end{example}

\begin{example}
	设$f(x)$在$(-\infty,+\infty)$上二次可微且有界,试证明存在$x_0\in(-\infty,+\infty)$,使得$f''(x_0)=0$。
\end{example}

\begin{example}
	设$f(x)$在$(-\infty,+\infty)$上具有二阶导数,且$f''(x)>0$,$\lim_{x\to+\infty}f'(x)=\alpha>0$,$\lim_{x\to-\infty}f'(x)=\beta<0$,又存在一点$x_0$,使得$f(x_0)<0$,试证明方程$f(x)=0$在$(-\infty,+\infty)$上有且只有两个实根。
\end{example}
